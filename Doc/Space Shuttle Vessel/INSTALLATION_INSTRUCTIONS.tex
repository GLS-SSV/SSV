\documentclass[Space_Shuttle_Vessel_Manual.tex]{subfiles} 
\begin{document}

\section{INSTALLATION INSTRUCTIONS}
\begin{multicols*}{2}

\subsection{Installation}
\noindent
\begin{enumerate}
\item Install Orbiter.

\item Install the required addons:\\
OrbiterSound 4.0 or 5.0 (\url{http://orbiter.dansteph.com/forum/index.php?page=download})\\
Antelope Valley scenery pack (\url{https://orbit.medphys.ucl.ac.uk/mirrors/orbiter_radio/tex_mirror.html})

\item Extract the SSV files into your Orbiter installation folder, overwriting any existing files.\\
\WARNING{The SSV installation overwrites the default Base.cfg and Earth.cfg files.}

\item Install the "SSV\_Font\_A" and "SSV\_Font\_B" fonts, located in the "<Orbiter installation>\textbackslash Install\textbackslash Space Shuttle Vessel" directory, by opening them and selecting "Install". After successful installation the files can be deleted.

\item The displays in SSV require the MFD resolution of 512 x 512 (Orbiter Launchpad $\rightarrow$ Extra $\rightarrow$ Instruments and panels $\rightarrow$ MFD parameter configuration $\rightarrow$ MFD texture size).
\end{enumerate}

\NOTE{If you encounter the error "msvcp140.dll is missing" you need to download the Microsoft Visual C++ Redistributable for Visual Studio 2017.}

\subsection{Optional addons}
For a better visual experience, using the D3D9 graphics client (\url{http://users.kymp.net/~p501474a/D3D9Client/}) is strongly recommended, although not required (minimum version R4.25). If using the D3D9 graphics client, the \textit{Disable near clip plane compatibility mode} option in the D3D9 Advanced Setup dialog (Orbiter Launchpad $\rightarrow$ Video $\rightarrow$ Advanced) should be checked.\\
For an accurate rendezvous profile simulation, it is recommended the installation of the excellent Shuttle FDO MFD (\url{https://github.com/indy91/Shuttle-FDO-MFD}) by indy91, which handles the ground based calculations required for rendezvous.\\
\end{multicols*}
\end{document}